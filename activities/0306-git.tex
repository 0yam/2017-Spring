\documentclass[a4paper, 11pt]{article}

\usepackage{fullpage}
\usepackage{hyperref}
\usepackage{amsthm}
\usepackage[numbers,sort&compress]{natbib}

\theoremstyle{definition}
\newtheorem{exercise}{Exercise}

\begin{document}
%%% Header starts
\noindent{\large\textbf{IS-521 Activity}\hfill
                \textbf{Git} \\
         {\phantom{} \hfill Due Date: March 8, 2017} \\
%%% Header ends

\section{Activity Overview}

The aim of this activity is to make sure everyone is familiar with
Git, GitHub, Markdown, and \LaTeX. As discussed in the first class,
all the course materials and your assignments will be posted on
GitHub. The benefits of using GitHub in education is already
known~\cite{feliciano:icse2016}. Some of them include:
\begin{enumerate}
  \item You don't need to prepare for your deliverables. Your latest
    assignments will be shared with course staffs.
  \item You can work on your assignment anywhere there is an internet
    connection.
  \item You will gain industry-relevant skills and practices.
  \item You will be able to contribute to the course contents with
    pull requests.
\end{enumerate}

\section{Exercises}

\begin{exercise}

  This is a warm-up exercise. Visit the Git Tutorial
  website~\cite{gittutorial} and finish all the stages.

\end{exercise}

\begin{exercise}

  Visit the \LaTeX~Tutorial website~\cite{latextutorial} and finish
  up to step 8. Make a commit for each step in order to show your
  progression.

\end{exercise}

\begin{exercise}

  Read the Markdown tutorial~\cite{mdtutorial}, and create a
  ``README.md'' file in your repository. The file should contain a
  brief description about your final \LaTeX~document, and it should
  start with the following line:
  \begin{verbatim}# is-521\end{verbatim}

\end{exercise}

\begin{exercise}

  In this final exercise, you will be intentionally creating a
  merge conflict, and you will eventually have to resolve it. Each
  student is given a private repository for this exercise. Please
  follow the steps below.
  \begin{enumerate}
    \item Create a branch called `test'.
    \item Make sure you are in the branch `master'.
    \item Make a modification on the ``README.md'' file. Specifically,
      you want to make the header to be capitalized:
      \begin{verbatim}# IS-521\end{verbatim}
    \item Make a commit.
    \item Now switch to the branch `test' you created.
    \item Make a modification on the ``README.md'' file. Specifically,
      you want to make the header to be:
      \begin{verbatim}# is-521: Git Exercise\end{verbatim}
    \item Make a commit. This time, you want to make the author name
      of the commit to be:
      \begin{verbatim}Unknown <unknown@unknown.com>\end{verbatim}.
      You can change the user name by modifying the file at
      ``.git/config`` or by using the command:
      \begin{verbatim}git commit --author=...\end{verbatim}.
    \item Switch back to the branch `master'.
    \item Merge the change you made in the `test' branch.
    \item Resolve the conflict. The final header would look like:
      \begin{verbatim}# IS-521: Git Exercise\end{verbatim}
    \item After resolving the conflict, push your modifications to
      your own assignment repository. Make sure your git repository
      includes the final merge commit.
    \item Annotate the last commit with a tag ``Activity-1'', and push
      the tag to the assignment repository.
  \end{enumerate}

\end{exercise}

\bibliography{references}
\bibliographystyle{plainnat}

\end{document}
