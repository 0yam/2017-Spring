\documentclass[a4paper, 11pt]{article}

\usepackage{fullpage}
\usepackage{hyperref}
\usepackage{amsthm}
\usepackage[numbers,sort&compress]{natbib}

\theoremstyle{definition}
\newtheorem{exercise}{Exercise}

\begin{document}
%%% Header starts
\noindent{\large\textbf{IS-521 Activity}\hfill
                \textbf{PGP} \\
         {\phantom{} \hfill Due Date: March 13, 2017 (before class)} \\
%%% Header ends

\section{Activity Overview}

In this activity, we learn how to securely encrypt your files and
personal emails with PGP. We are going to use GnuPG, an open-source
implementation for PGP.

\section{PGP Key-Signing Party}

The goal of this exercise is twofold: (1) students get to know each
other; (2) students establish the web of trust using GPG.
\begin{enumerate}
  \item Create a PGP key of your own with GPG.
  \item Push your public key to our web page (by opening a pull
    request).
  \item Meet your colleagues and exchange your public keys.
  \item A student who gets the most signatures will get an extra point
    (+1).
  \item You can also ask course staffs to sign your key.
\end{enumerate}

\section{Ethical Conduct Agreement}

Git allows us to sign tags and commits with your PGP key. Read this
web page~\cite{gitgpg} and configure your PGP key with Git, and follow
the steps below.

\begin{enumerate}
  \item Fork the repository at
    \url{https://github.com/KAIST-IS521/Agreement}.
  \item Modify the \texttt{Agreement.md} file appropriately: correct
    the date and the name.
  \item Commit and push your modification.
  \item Tag your commit with ``Agreement''.
  \item Sign the tag, and push it.
  \item Open a pull request.
\end{enumerate}

\section{Sending an encrypted E-mail}

Send an email that is signed by your private key and encrypted by
professor's public key. The content of your email should contain the
following line:
\begin{verbatim}
Your Name, Your GitHub ID, Your Student ID
\end{verbatim}

\bibliography{references}
\bibliographystyle{plainnat}

\end{document}
